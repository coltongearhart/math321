\documentclass{article}
\usepackage{style-notes}

\newcounter{lecnum} 	% define counter for lecture number
\renewcommand{\thepage}{\thelecnum-\arabic{page}}	% define how page number is displayed (< lecture number > - < page number >)

% define lecture header and page numbers
% NOTE: to call use \lecture{< Lecture # >, < Lecture name >, < Chapter # >, < Chapter name >, < Section #s >}
\newcommand{\lecture}[5]{

    % define headers for first page
    \thispagestyle{empty} % removes page number from page where call is made

    \setcounter{lecnum}{#1}		% set lecture counter to argument specified

    % define header box
    \begin{center}
    \framebox{
      \vbox{\vspace{2mm}
    \hbox to 6.28in {\textbf{MATH 320: Probability} \hfill}
       \vspace{4mm}
       \hbox to 6.28in {{\hfill \Large{Lecture #1: #2} \hfill}}
       \vspace{2mm}
       \hbox to 6.28in {\hfill Chapters #3: #4 \small{(#5)}}
      \vspace{2mm}}
    }
    \end{center}
    \vspace{4mm}
    
    % define headers for subsequent pages
    \fancyhead[LE]{\textit{#2} \hfill \thepage} 		% set left header for even pages
    \fancyhead[RO]{\hfill \thepage}		% set right header for odd pages

}

% define macros (/shortcuts)
\newcommand{\bu}[1]{\textbf{\ul{#1}}}				% shortcut bold and underline text in one command
\newcommand{\blankul}[1]{\rule[-1.5mm]{#1}{0.15mm}}	% shortcut for blank underline, where the only option needed to specify is length (# and units (cm or mm, etc.)))
\newcommand{\vecinf}[1]{#1_1, #1_{2}, \ldots}		% define another vector of the form X_1, X_2, ....
\newcommand{\integral}[4]{\displaystyle \int_{#1}^{#2} #3 \,\mathrm{d} #4}		% shortcut for large integral with limits and appending formatted dx (variable x)


% NOTES on what didn't cover
% Theory lecture 5 -> formal notation about range X = {s in S: X(s) = x} stuff

\begin{document}

\lecture{7}{Random Variables}{2 and 3}{Distributions}{2.1 and 3.1}

Why do we study statistics?\bigskip
\begin{itemize}
    \item The main purpose of studying statistics is because we want to study experiments and their outcomes.
    \item We want to analyze data from experiments numerically. But, outcomes are not always quantitative.
    \begin{itemize}
        \item So we have to assign numbers to outcomes. Thus, random variables connect outcomes to numbers.
        \item The advantage using random variables is that they are easily summarized.
    \end{itemize}
    \item Intuitive definition: A \textbf{random variable} is a numerical quantity whose value depends on chance.
\end{itemize}\bigskip

Types of random variables (RVs)\bigskip
\begin{itemize}
    \item Examples: Determine if each describes a RV.
    \item[] i.e. Is the outcome (a) is a number? (b) depends on chance?
    \begin{enumerate}
        \item You are tossing a coin twice and will bet on the number of heads.
        \item You go to Las Vegas and begin to put quarters in a slot machine. Let $X$ be the number of quarters you play in order to first win of any amount.
        \item You are tossing a coin twice and will bet on specific outcomes such as $HT$.
        \item A resident of Muncie is selected at random, and their height is measured.
    \end{enumerate}\bigskip
    \item Similar to sample spaces, there are different kinds of random variables.
    \item[] \textbf{This will be a very important distinction to make at the start of every single problem for the rest of the course.}
    \item Random variables can be discrete (only distinct values are possible) or continuous (measured on a continuous scale).
    \begin{itemize}
        \item When classifying a random variable as discrete or continuous, we are really just identifying the kind of mathematical model we will use. 
        \item Calculus-based mathematics is the most efficient way to analyze a random variable such as heights (which we may only measure as discrete to a certain precision).
    \end{itemize}
\end{itemize}\bigskip

Definitions and notation\bigskip
\begin{itemize}
    \item Functions \textit{map} the input (domain, support) to the output (range).\bigskip
    \item Our general definition of probability was a way to assign a probability $P(A)$ to any event $A$ where all the axioms needed to be satisfied. This, more formally, is a function.    
    \item A \textbf{random variable} is a function from a sample space $S$ into real numbers.\vspace{90pt}\\
    \begin{tabular}{rll}
         & \ul{Random variable} \hspace{100pt} & \ul{Probability} \\\\
         Input: & &\\\\
         Output: & &\\\\
         Maps: & &
     \end{tabular}

    \item Notation: We will use uppercase letters, such as $X$, $Y$, $Z$, $\ldots$ to denote a random variable and lowercase letters, such as $x$, $y$, $z$, $\ldots$ to denote a particular value that a random variable may assume.\bigskip
    \item Definition: The set of possible values of $X$ is the \textbf{range} of $X$, ${\cal X}$.
    \item Summary of notation:
    \begin{itemize}
        \item $X = $ Random variable.
        \item $x_i = $ Individual values of $X$.
        \item ${\cal X} = $ Range of $X$ $\rightarrow$ set of all $x_i = \{\vecinf{x}\}$ or $[x_a, x_b]$
    \end{itemize}\bigskip
    \item It is important to know the distinction between the outcomes in an experiment (sample space) and the range.
    \item Examples:
    \begin{enumerate}
        \item Toss three fair coins and observe the results. Let $X$ equal the number of heads obtained.     
        \begin{enumerate}
            \item What is the sample space and range of $X$?\vspace{30pt}
            \item Show the connection between $S$ and $X$.\vspace{70pt}
        \end{enumerate}
        \item Let $X$ be the time to failure for a machine part. Find the range.\vspace{30pt}
        \item You are waiting for the bus to arrive. If it arrives in under 5 minutes, you will get on the bus. If not, you will walk to your destination.
    \item[] Let $X$ be the random variable such that $X = 1$ if you get on the bus and $X = 0$ if you walk. Is $X$ a continuous or discrete random variable?\vspace{70pt}
    \end{enumerate}
    \item Types of random variables definitions
    \item[] $X$ is a \textbf{discrete random variable} if the \blankul{2cm} is a finite or countable set.
    \item[] $X$ is a \textbf{continuous random variable} if the \blankul{2cm} is an interval (or union of intervals) on the real number line.
\end{itemize}\bigskip

Connection between random variables and probability\bigskip
\begin{itemize}
    \item We would like use random variables to express events, because we can calculate probabilities of events.
    \item Notation: $\{X = x\}$ is the set of \blankul{2cm} in the sample space assigned the value $x$ by the random variable $X$.
    \item[] $X = x$ means the random variable $X$ was realized with a specific value $x$.
    \item[] So it is an \blankul{2cm}. As a result, we can compute the probability of $\{X = x\}$.    \item Notation: We used to have events like $A \cap B$ or now $\{X = x\}$ in $P(\cdot)$, but we will now use $P(X = x)$ for simplicity.
\bigskip
    \item[] Example: Continuing the previous three coin toss scenario, find the following events and their probabilities:
    \item[] $\{X = 1\} = $\bigskip
    \item[] $\{X = 3\} = $\bigskip
\end{itemize}





\end{document}