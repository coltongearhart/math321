\documentclass{article}
\usepackage{style-syllabus}

\begin{document}

\begin{center}
\Huge{MATH 321: Mathematical Statistics}

\large{Spring 2024, Ball State University}
\end{center}

\bigskip\bigskip

\textbf{\large Course Information} (MATH 321-1)\medskip

\begin{tabular}{ll}
    \textbf{Class}: & MTRF 11:00 - 11:50 AM, Robert Bell 115\\\\
    \textbf{Instructor}: & Colton Gearhart (email: colton.gearhart@bsu.edu) \\
     & You can expect a response to emails within 24 - 48 hours \\\\
    \textbf{Office Hours}: & MF 1:00 - 2:00 PM or by appointment, Office RB 411 \\
     & \textbf{Please make appointment for office hours}
\end{tabular}\bigskip

\textbf{\large Course Materials}\medskip

\begin{tabular}{p{0.2\linewidth}p{0.75\linewidth}}
    \textbf{Textbook}: & Probability and Statistical Inference, 10th ed., Hogg, Tanis, and Zimmerman, Pearson 2020 (Recommended)\\\\
    \textbf{Calculator}: & TI-30XS MultiView or TI-84 Graphing Calculator\\\\
    \textbf{Course Website} & Assignments, solutions and some other class materials will be posted on Canvas
\end{tabular}\bigskip

\textbf{\large Course Description}\medskip

Random sampling, statistical inference, and sampling distributions, point and interval estimation, matching moments, maximum likelihood, mean square error, consistency, efficiency, uniformly minimum-variance unbiased estimator (UMVUE), Neyman-Pearson Lemma, Likelihood ratio tests, classical tests of significance, goodness-of-fit, contingency tables, correlation, regression, nonparametric methods, Bayesian methods.

Prerequisites: C- or better in MATH 320 or permission of the department chairperson.\\
4 Credit hours (4 Lecture hours).\bigskip

\textbf{\large Course Objectives}\medskip

Students will become familiar with both the theory and applications of the three steps a statistician faces - model selection, model verification, and model interpretation. Students will practice selecting the appropriate model to fit a real-life problem. Students will analyze and evaluate the reasonableness of the model. Students will draw appropriate conclusions from the model to solve the proposed problem.\bigskip

\textbf{\large Course Rationale}\medskip

Mathematical statistics is the study of how to deal with data by means of probability models. It grew out of methods for treating data that were obtained by some repetitive operation such as those encountered in games of chance and in industrial processes. These methods soon found application in such diverse fields as medical research, insurance, marketing, agriculture, chemistry, and in industrial experimentation. This course is, therefore, a required course for statistics and actuarial science majors and is an excellent course for those who may need in their work statistical methods at a reasonably sophisticated level.\bigskip

\newpage

\textbf{\large Course Content}\medskip

The following is a list of subjects to be taught in the course given sufficient time:

\begin{enumerate}
    \item Estimation
    \item[]Maximum Likelihood Estimation, Properties of Estimators, Confidence Intervals for Means, Confidence Intervals for Variances, Confidence Intervals for Proportions, Sample Size, Sufficient Statistics, Chebyshev's Inequality.
    \item Tests of Statistical Hypotheses
    \item[] Critical Region, Type I and type II Errors, Power, Best Critical Regions, Neyman-Pearson Theorem, Likelihood Ratio Tests, Tests About Means and Proportions, Test About Variance, Tests About Difference of Means.
    \item Nonparametric Methods
    \item[] Order Statistics, Confidence Intervals for Percentiles, Binomial Tests for Percentiles, Wilcoxon Test, Two-Sample Distribution-Free Tests, Chi-square Tests for Models, Testing Probabilistic Models, Comparisons of Several Distributions, Contingency Tables.
    \item Linear Statistical Models
    \item[] Simple Regression, Tests of Equality of Means, Two-Factor Analysis of Variance.
    \item Bayesian Decision Theory
    \item[] Compound distributions, Decision Theory, Bayesian Methods.
\end{enumerate}\bigskip

\textbf{\large In-Class Activities}\medskip

Throughout the semester, there will be in-class activities worth a total of 15\% of your grade. The lowest score of these activities will be dropped. The in-class tasks will be mainly designed to practice what you studied recently.

If you miss class when there is an in-class activity, you will receive a zero unless there is a legitimate reason pre-approved by me.\bigskip

\textbf{\large Homework}\medskip

Homework assignments will account for 20\% of the course grade.\medskip
\begin{itemize}
    \item The lowest homework assignment score will be dropped.
    \item Any assignment turned in within one week after the deadline may be given a 50\% grade reduction. After that, it will not be accepted. An exception to this reduction policy will be considered for legitimate circumstances that are presented to me (via e-mail or in person) \textit{before the due date}. 
    \item Submissions must be neat and stapled.
    \item Unorganized and/or illegible work will be considered incorrect.
\end{itemize}\medskip

\ul{Collaboration}: Feel free to work together, but the work you turn in should be your own. For example, if you solve a problem with a friend and the two of you write down identical solutions, you both will be in violation of this policy. Instead, after solving the problem together, you might re-solve it on your own so that it is evident the work is yours. If your work is not clearly independent of others', a score of 0 may be awarded for the entire offending assignment.\medskip

\newpage

\ul{Requesting help}: The homework is essential to success in this class because it forms the foundation of knowledge and problem-solving skill upon which exam success can be built. I am glad to help you with the homework, but it is in your own best interest to work on a problem for awhile before you come for help, even if you are stuck on it.\medskip

\ul{Showing work}: In this class, correct answers are typically not enough to get full credit. Any work that you turn in (whether homework, in-class assignments, tests or final) must demonstrate the process by which the solution was obtained. This means that if you write down the correct answer with no supporting work you may receive partial credit or no credit at all.\bigskip

\textbf{\large Tests and Final}\medskip

There will be three Tests that together account for 45\% of the total grade (15\% per Test), each will be held during the entire class period.

At the end of the semester, there will be a comprehensive Final worth 20\% of the total grade. It will be held on Thursday, May 2 from 9:45 AM to 11:45 AM.\bigskip

\textbf{\large Grading}\medskip

Your final grade will be comprised of the following elements:\bigskip\\
\begin{tabular}{ll}
    \textbf{In-Class Assignments} & 15\%\\
    \textbf{Homework} & 20\%\\
    \textbf{Tests} & 45\% (3 Tests, each worth 15\%)\\
    \textbf{Final} & 20\% 
\end{tabular}\medskip

Letter grades:\bigskip\\
\begin{tabular}{ll | ll | ll | ll | ll}
    \hline
    $[93 - 100]$ & A & $[87 - 90)$ & B+ & $[77 - 80)$ & C+ & $[67 - 70)$ & D+ & $< 60$ & F\\
    $[90 - 93)$ & A-  & $[83 - 87)$ & B & $[73 - 77)$ & C & $[63 - 67)$ & D & \\
     & & $[80 - 83)$ & B- & $[70 - 73)$ & C- & $[60 - 63)$ & D- & \\
    \hline
\end{tabular}

I may lower the grade thresholds, but will not raise them.\bigskip

\textbf{\large Attendance Policy}\medskip

The pace of this class is such that it will not be advisable to miss any sessions. If you know you will be absent, please inform me in advance. When you are absent, it will be your responsibility to look on Canvas for the uploaded notes and announcements.

I really want to encourage you to ASK QUESTIONS and take part in the lectures! This is part of the learning process and benefits others in the class who may have the same questions.\bigskip

\textbf{\large Important Dates}\medskip

\begin{tabular}{ll}
    January 8, Monday & Classes begin\\
    January 15, Monday & MLK Day, no class\\
    March 4 -- 8, Monday -- Friday & Spring Break, no class\\
    March 21, Thursday & \textbf{Last day to drop with "W"}\\
    April 29, Monday & Last day of class\\
    May 2, Thursday & Final Exam at 9:45 AM
\end{tabular}\bigskip

\newpage

\textbf{\large Withdrawal Statement}\medskip

The course withdrawal period ends \textbf{Thursday, March 21, 2024 at 5:00 PM}. Before this date, students can elect to receive a "W" for the course by completing and submitting the proper form. The instructor's permission is not required. For details, see \href{https://www.bsu.edu/about/administrativeoffices/registrar/registration-activities/withdraw-from-classes}{here} as well as Degree Requirements and Time Limits in the current Undergraduate Catalog OR Withdrawal Procedures in the current graduate catalog.\bigskip

\textbf{\large Disability Statement}\medskip

Do not hesitate to contact me with any questions or concerns. If you need course adaptations or accommodations because of a disability, please contact me as soon as possible. \href{https://www.bsu.edu/about/administrativeoffices/disability-services}{The Office of Disability Services} coordinates services for students with disabilities; documentation of a disability needs to be on file in that office before any accommodations can be provided. Disability Services can be contacted at 765-285-5293 or dsd@bsu.edu.

If you are experiencing mental health concerns, telehealth services are available ? here is a link to the \href{https://www.bsu.edu/campuslife/counseling-center}{Counseling Center website}.\bigskip

\textbf{\large Diversity Statement}\medskip

Ball State University aspires to be a university that attracts and retains a diverse faculty, staff, and student body. We are committed to ensuring that all members of the community are welcome, through valuing the various experiences and worldviews represented at Ball State and among those we serve. We promote a culture of respect and civil discourse as expressed in our \href{https://www.bsu.edu/about/beneficence}{Pledge Beneficence} and through university resources found \href{http://bsu.edu/campuslife/multiculturalcenter}{here}.\bigskip

\textbf{\large Important Links}\medskip

\href{https://www.bsu.edu/about/administrativeoffices/vice-provost/student-services/academic-integrity}{Student Academic Ethics Policy}

\href{https://www.bsu.edu/about/administrativeoffices/student-conduct/policiesandprocedures}{Code of Student Rights and Responsibilities}\bigskip

\center \textit{Syllabus is subject to change}

\end{document}