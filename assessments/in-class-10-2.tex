\documentclass{article}
\usepackage{style-assessments}

% define macros (/shortcuts)

\begin{document}

\hspace{375pt}Name:

\begin{center}
{\Huge MATH 320: In-Class 10-2}

\end{center}
\bigskip\bigskip

\begin{enumerate}
    \item Consider the experiment of drawing from a deck of cards with replacement.% Actex 5-27
    \begin{enumerate}
        \item What is the probability that the third heart appears on the tenth draw?\vspace{100pt}
        \item What is the probability that the third heart appears before the seventh draw?\vspace{100pt}
        \item What is the mean number of cards drawn to get the fifth red card?\vspace{60pt}
    \end{enumerate}
    
    \item 
    \begin{enumerate} % Acted 5-12 (with original additions)
        \item In a hospital there are 120 patients, 10 of whom have a particular disease. If a doctor is assigned 6 patients, what is the probability they receive more than 2 of these patients?\vspace{100pt}
        \item In a different hospital there are 30 patients, 13 of whom have a particular disease. If another doctor is assigned 22 patients, what is the probability they receive no more than 8 of these patients?\vspace{100pt}
    \end{enumerate}
    
    \item An insurance company has 5,000 policyholders who have had policies for at least 10 years. Over this period there have been a total of 12,200 claims on these policies. Assuming a Poisson distribution for these claims, answer each of the following questions:% Actex 5-19
    \begin{enumerate}
        \item What is $\lambda$, the average number of claims per policy per year?\vspace{40pt}
        \item What is the probability that a policyholder will file less than 2 claims in a year?\vspace{100pt}
        \item If all claims are for \$1,000, what is the mean and variance for the claim amount for a policyholder in a year?\vspace{100pt}
    \end{enumerate}
    
    \item A company prices its hurricane insurance using the following assumptions:% Actex 5-36
    \begin{enumerate}[(i)]
        \item In any calendar year, there can be at most one hurricane.
        \item In any calendar year, the probability of a hurricane in 0.05.
        \item The number of hurricanes in any calendar year is independent of the number of hurricanes in any other calendar year.
    \end{enumerate}
    \item[] Using the company's assumptions, calculate the probability that there are fewer than 3 hurricanes in a 20-year period.\vspace{100pt}
\end{enumerate}


\end{document}