\documentclass{article}
\usepackage{style-assessments}

% define macros (/shortcuts)
\newcommand{\blankul}[1]{\rule[-1.5mm]{#1}{0.15mm}}	% shortcut for blank underline, where the only option needed to specify is length (# and units (cm or mm, etc.)))
\newcommand{\vecn}[2]{#1_1, \ldots, #1_{#2}}	% define vector (without parentheses, so when writing out in like a definition) of the form X_1, ..., X_n, where X and n are variable. NOTE: to call use $\vecn{X}{k}$
\newcommand{\follow}[1]{\sim \text{#1}\,}		% shortcut for ~ 'Named dist ' in normal font with space before parameters would go
\newcommand{\followsp}[2]{\overset{#1}\sim \text{#2}\,}		% (followsp is short for 'follow special') shortcut that can be used for iid or ind ~ 'Named dist ' in normal font with space before parameters would go
\newcommand{\chisq}{\raisebox{2pt}{$\chi^2$}}		% shortcut for chi-square distribution (better formatted chi letter in math mode with square added)
\newcommand{\ind}{\perp \!\!\! \perp}			% define independence symbol (it basically makes two orthogonal symbols very close to each other. The number of \! controls the space between each of the orthogonal symbols)

\begin{document}

\hspace{375pt}Name:

\begin{center}
{\Huge MATH 321: Homework 1}
\end{center}

\bigskip\bigskip

{\large \textbf{Due} \blankul{4cm}: Turn in a hard copy, neat and stapled.}\bigskip

% problem types summary
% 1) normal sample mean probability, population probability, using normal prob in a binomial dist prob, sample variance (chi-square) prob
% 2) finding distributions of F and then probabilities
% 3) finding F probabilities
% 4) R simulation

\begin{enumerate}

    \item Entrance exam scores of students at a certain college, $X$, are normally distributed with \\ $X \follow{N}(\mu = 1300, \sigma^2 = 2500)$. Suppose an admissions officer selects a random sample of 30 students at this college and checks their entrance exam scores.%Lazar HW 2 AQ 6
    \begin{enumerate}
        \item If $\bar{X}$ is the sample mean of the 30 exam scores, find the distribution of $\bar{X}$.
        \item Find $P(1290 \le \bar{X} \le 1310)$.%part b
        \item Find $P(X \le 1230)$.%Original addition (to make next part easier)
        \item Let $Y$ be the number of random variables (exam scores) in the sample that have values of at most 1230. Find the probability that less than 5 of the random variables in the sample have scores of at most 1230, that is, $P(Y < 5)$.%like part a, but broken up like AQ 8 III and IV
        \item[] \textit{HINT: Think of $P(X \le 1230)$ as a success probability, and the result of checking this is for each random variable is either a success or failure.} 
        \item Let $S^2$ be the sample variance of the 30 exam scores. Find $P(S^2 > 2000)$.%like part c, but simpler
        \item[] \textit{HINT: Think about how to get the random variable of interest to follow a distribution we know.} 
    \end{enumerate}\bigskip

    \item Let $X_1 \follow{\chisq}(10)$ and $X_2 \follow{\chisq}(13)$ and $X_1 \ind X_2$.%Original question
    \begin{enumerate}
        \item Let $\displaystyle Y_1 = \frac{X_1 / 10}{X_2 / 13}$. Find the distribution of $Y_1$ and $P(Y_1 > 5)$.
        \item Let $Y_2 = 1/Y_1$. Find the distribution of $Y_2$ and the $IQR$ of $Y_2$.
    \end{enumerate}\bigskip
        
    \item Suppose we take independent random samples of sizes $n_1 = 6$ and $n_2 = 10$ from two normal populations with equal population variances ($\sigma_1^2 = \sigma_2^2)$. Let $S_1^2$ and $S_2^2$ be the sample variances from populations 1 and 2, respectively.%Math stats with apps 7.27
    \item[] Find $P(S_1^2 / S_2^2 > 2)$.
    
    \newpage
    
    \item \textbf{R simulation}: We are going to simulate a sampling distribution for a statistic and estimate some probabilities using empirical methods.
    \item[] \textbf{Guidelines}: Complete each of the following steps for the following pairs of distribution and statistic: 1) Gamma distribution and median and 2) Your choice (be creative!). For example, Binomial / sample variance or $\chisq$ / minimum.
    \begin{enumerate}
        \item Plot the population distribution of interest that you will be sampling from.
        \item Generate $i = 10,000$ random samples of size $n = 30$ from the population distribution.
        \item Calculate the sample statistic $Y = T(\vecn{X}{30})$ for each of the random samples.
        \item Plot a histogram of the simulated sampling distribution of $Y$.
        \item Calculate $\hat{\mu}_Y$ and $\hat{\sigma}_Y$, the estimated mean and standard deviation of the sampling distribution of the sample statistic based on the simulated results, respectively.
        \item Calculate the estimated probability the sample statistic is within two standard deviations of its mean: \\$P(\hat{\mu}_Y - 2\hat{\sigma}_Y < Y < \hat{\mu}_Y + 2\hat{\sigma}_Y)$.
    \end{enumerate}
    \item[] \textbf{Restrictions}: Do not use the normal distribution or the sample mean $\bar{X}$.
    \item[] \textbf{Submission}: This problem will be worth 15 of the 30 points. Please submit your completed version of the starter .qmd file on canvas and rendered .html file.
\end{enumerate}\vspace{100pt}

Select answers\bigskip
\begin{enumerate}
    \item 
    \begin{enumerate}
        \item 
        \item $P(1290 \le \bar{X} \le 1310) \approx 0.7267$
        \item $P(X \le 1230) \approx 0.0808$
        \item $P(Y < 5) \approx 0.9097$
        \item $P(S^2 > 2000) \approx 0.7673$
    \end{enumerate}
    
    \item 
    \begin{enumerate}
        \item $P(Y_1 > 5) \approx 0.0042$
        \item $IQR \text{ of } Y_2 \approx 0.8639$
    \end{enumerate}
    
    \item $P(S_1^2 / S_2^2 > 2) \approx 0.1727$
    
    \item 
            
\end{enumerate}

\end{document}