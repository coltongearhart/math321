\documentclass{article}
\usepackage{style-assessments}

% define macros (/shortcuts)
\newcommand{\vecn}[2]{#1_1, \ldots, #1_{#2}}	% define vector (without parentheses, so when writing out in like a definition) of the form X_1, ..., X_n, where X and n are variable. NOTE: to call use $\vecn{X}{k}$
\newcommand{\followsp}[2]{\overset{#1}\sim \text{#2}\,}		% (followsp is short for 'follow special') shortcut that can be used for iid or ind ~ 'Named dist ' in normal font with space before parameters would go



\begin{document}

\hspace{375pt}Name:

\begin{center}
{\Huge MATH 321: In-Class 6}
\end{center}

\bigskip\bigskip

% problem types summary
% 1) conditions for CI for p, two sided interval, one sided lower and upper intervals
% 2) difference in two proportions two sided interval
% 3) two sided CI for p
% 4) difference in two proportions two sided interval
% 5) several two sided intervals for p, changing different variables for the CI and seeing results



\begin{enumerate}
    \item A NatGeo Poll interviewed 1200 hiking enthusiasts and asked ``Are you more afraid of spiders or snakes??'' Out of the 1200 people, 768 responded ``Ewww, snakes...''
    \begin{enumerate}%CSCC 1450 UPDATED LU 7 - Confidence Intervals and Sample Size for Proportions - Days 1 and 2
        \item Check the conditions for a confidence interval for the true proportion $p$ of hikers who are more afraid of snakes.\vspace{20pt}
        \item Calculate a 95\% confidence interval for $p$.\vspace{60pt}
        \item Calculate a 90\% lower-bound confidence interval for $p$ and a 90\% upper-bound confidence interval for $p$.\vspace{60pt}% Extra addition
    \end{enumerate}
    
    \item The poll from (1) also asked 1100 climbers the same question. 662 of the 1100 climbers, responded ``Ewww, snakes....''%CSCC 1450 UPDATED LU 9 - Inferences from Two Samples
    \item[] Calculate a 92\% confidence interval for the difference in proportion of climbers vs hikers who are more afraid of snakes than spiders. State the conclusion as well.\vspace{120pt}
    
    \item 15 out of 23 people from a random sample said their National Championship team is still remaining in their NCAA March Madness Bracket.%CSCC 1450 UPDATED LU 7 - Confidence Intervals and Sample Size for Proportions - Days 1 and 2
    \item[] Calculate a 85\% confidence interval for the true proportion $p$ of brackets that still have their National Championship team remaining.\vspace{70pt}\newpage
 
    \item For a comparison of the rates of defectives produced by two assembly lines, independent random samples of 100 items were selected from each line. Line A yielded 18 defectives in the sample, and line B yielded 12 defectives.
    \item[] Find a 98\% confidence interval for the true difference in proportions of defectives for the two lines AND state a conclusion if one line produces a higher proportion of defectives than the other.\vspace{120pt}% Math stats with apps problem 8.65
    
    \item From a random sample 500 people, 64\% said they prefer to vacation at the beach compared to the mountains.%CSCC 1450 UPDATED LU 7 - Confidence Intervals and Sample Size for Proportions - Days 1 and 2
    \begin{enumerate}
        \item Calculate a 93\% confidence interval for the true proportion $p$ of people who prefer beaches over mountains for vacation.\vspace{70pt}
        \item If the sample size was increased to 600 people and all else remains constant, what will happen to the new confidence interval? Calculate this new interval.\vspace{70pt}
        \item If the confidence level for the interval from (a) changed to 90\% confident, what will happen to the new confidence interval? Calculate this new interval.\vspace{70pt}
    \end{enumerate}
        
\end{enumerate}


\end{document}