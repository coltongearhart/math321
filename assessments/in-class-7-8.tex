\documentclass{article}
\usepackage{style-assessments}

% define macros (/shortcuts)
\newcommand{\e}{\mathrm{e}}		% shortcut for non-italic e in math mode

\begin{document}

\hspace{375pt}Name:

\begin{center}
{\Huge MATH 320: In-Class 7-8}

\end{center}

\bigskip\bigskip


Answer all questions. Show your work where necessary.\bigskip

% problem types summary
% 1) Random variables
% 2) Writing and plotting pmf and cdf of discrete RV
% 3) Finding discrete pmf from piecewise cdf
% 4) Find c to make a valid pdf
% 5) Exponential cdf prob, find pdf from cdf, find left prob using pdf and then complement prob, find interval prob using both pdf and cdf
% 6) Find probs of straight line density and find cdf


\begin{enumerate}
    \item Toss a fair coin until the first head occurs. Let $X$ be the number of tosses until the first head occurs.%Theory In-class 3
    \begin{enumerate}
        \item Find the sample space $S$ and the range of $X$, $\cal{X}$.\vspace{30pt}
        \item Define the random variable $X$ using a figure. It should include $S$, $\cal{X}$ and arrows.\vspace{60pt}
    \end{enumerate}
    
    \item Consider the experiment of tossing a fair coin three times. Define the random variable $X$ as the number of heads observed.% Theory lecture 6 page 2 example
    \item[] (a) Write the pmf $f_X(x)$ as a piecewise function. \hspace{20pt} (b) Write the cdf $F_X(x)$ as a piecewise function.\vspace{90pt}
    \item[] (c) Plot both the pmf and cdf.\vspace{100pt}
    
    \item A random variable $X$ has the cdf: % Theory lecture 6 page 3 example
    \[
    F_X(x) =  P(X \le x) = 
        \left\{
        \begin{array}{ll}
             0 & x < -1\\
             0.2 & -1 \le x < 0\\
             0.7 & 0 \le x < 1\\
             1 & 1 \le x \\
        \end{array}
        \right.
    \]
    \begin{enumerate}
        \item Is $X$ a discrete or continuous random variable? Why?\vspace{30pt}
        \item Find $P(-1 < X \le 1)$ \hspace{50pt} (c) Find the pmf $f_X(x)$.\vspace{50pt}
    \end{enumerate}
        
    \item Let $\displaystyle f(x) = \frac{1}{6} (x+1), \quad 1 < x < c$. Find the constant $c$ so that $f(x)$ is a valid pdf.\vspace{120pt}%Lazar HW 4 Additional question 5 (just a part of it)
    
    \item Let $F_X(x) = 1 - \e^{-5x}, \quad 0 \le x < \infty$.% Original question
    \begin{enumerate}
        \item Find $P(X < 3)$.\vspace{40pt}
        \item Find the pdf $f_X(x)$.\vspace{60pt}
        \item Find $P(X < 3)$ using the pdf AND THEN find $P(X \ge 3)$.\vspace{100pt}
        \item Find $P( 1 \le X \le 5)$ using the pdf AND THEN using the cdf.\vspace{80pt}
    \end{enumerate}
    
    \item Let $f(x) = 1.5x + 0.25, \quad 0 \le x \le 1$.% Actex 7-1 (the equation), original questions
    \begin{enumerate}
        \item Find $P(X \le 0.5)$ and $P(X \ge 0.75)$ using areas of shapes.\vspace{60pt}
        \item Find the cdf $F(x)$.\vspace{60pt}
    \end{enumerate}

    
\end{enumerate}

\end{document}